%%%%%%%%%%%%%%%%%%%%%%%%%%%%%%%%%%%%%%%%%%%%%%%%%%%%%%%%%%%%%%%%%%%%%%%%
%%% documentclass and packages
%%%%%%%%%%%%%%%%%%%%%%%%%%%%%%%%%%%%%%%%%%%%%%%%%%%%%%%%%%%%%%%%%%%%%%%%
\RequirePackage{atbegshi}           % workaround for newer PGF versions
%\documentclass[hyperref={pdfpagelabels=false}]{beamer}
\documentclass{beamer}
% https://sourceforge.net/tracker/index.php?func=detail&aid=1848912&group_id=92412&atid=600660
\usepackage{lmodern}
\usepackage[T1]{fontenc}
\usepackage[utf8]{inputenc}
\usepackage{textcomp}
\usepackage[ngerman]{babel}
\usepackage[babel,english=american,german=guillemets]{csquotes}	% french
\usepackage{microtype}

% colors for listings
\definecolor{lightergray}{gray}{.95}
\definecolor{darkblue}{rgb}{0,0,0.5}
\definecolor{darkgreen}{rgb}{0,0.5,0}
\definecolor{darkred}{rgb}{0.5,0,0}
\definecolor{darkerblue}{rgb}{0,0,0.4}
\definecolor{darkergreen}{rgb}{0,0.4,0}
\definecolor{darkerred}{rgb}{0.4,0,0}

%\usepackage{listings}
%\lstloadlanguages{HTML,XML}
%\lstset{
%    basicstyle=\ttfamily\small\mdseries,
%    keywordstyle=\bfseries\color{darkblue},
%    identifierstyle=,
%    commentstyle=\color{darkgray},
%    stringstyle=\itshape\color{darkred},
%    frame=none,
%    showstringspaces=false,
%    tabsize=4,
%    backgroundcolor=\color{lightergray},
%}

%%%%%%%%%%%%%%%%%%%%%%%%%%%%%%%%%%%%%%%%%%%%%%%%%%%%%%%%%%%%%%%%%%%%%%%%
%%% preparations for beamer
%%%%%%%%%%%%%%%%%%%%%%%%%%%%%%%%%%%%%%%%%%%%%%%%%%%%%%%%%%%%%%%%%%%%%%%%
\useinnertheme{default}
\useoutertheme{infolines}
%\usecolortheme[rgb={0.28,0.37,0.52}]{structure}
\usecolortheme[rgb={0.18,0.23,0.33}]{structure}
%\usecolortheme{beaver}
\usefonttheme{structurebold}

%%% Ränder vergrößern, sieht dann nicht so an den Rand gequetscht aus
\setbeamersize{text margin left=0.5cm}
\setbeamersize{text margin right=0.5cm}

%%%%%%%%%%%%%%%%%%%%%%%%%%%%%%%%%%%%%%%%%%%%%%%%%%%%%%%%%%%%%%%%%%%%%%%%
%%% images
%%%%%%%%%%%%%%%%%%%%%%%%%%%%%%%%%%%%%%%%%%%%%%%%%%%%%%%%%%%%%%%%%%%%%%%%
%\pgfdeclareimage[width=\textwidth]{lespockydeimpressum}{lespocky.de_impressum}
%\pgfdeclareimage[width=0.6\textwidth]{psituxsmileybeispiel}{psi_tux_smiley_beispiel}

%%%%%%%%%%%%%%%%%%%%%%%%%%%%%%%%%%%%%%%%%%%%%%%%%%%%%%%%%%%%%%%%%%%%%%%%
%%% title, author, date
%%%%%%%%%%%%%%%%%%%%%%%%%%%%%%%%%%%%%%%%%%%%%%%%%%%%%%%%%%%%%%%%%%%%%%%%
\title[PGP]{Einführung in PGP}
\subtitle{Auch: Einführung in OpenPGP und GnuPG}
\author{Alexander Dahl}
\institute[lespocky.de]{\url{http://www.lespocky.de/}}
\date{2017-09-25}
%\subject{subj}
\keywords{PGP, OpenPGP, GnuPG}

%%%%%%%%%%%%%%%%%%%%%%%%%%%%%%%%%%%%%%%%%%%%%%%%%%%%%%%%%%%%%%%%%%%%%%%%
%%% document
%%%%%%%%%%%%%%%%%%%%%%%%%%%%%%%%%%%%%%%%%%%%%%%%%%%%%%%%%%%%%%%%%%%%%%%%
\begin{document}

\begin{frame}
	\titlepage
\end{frame}

\begin{frame}{PGP}
    \begin{itemize}
        \item Phil Zimmermann (1991)
        \item Public-Key-Verfahren: asymmetrische Verschlüsselung
        \item Öffentlicher Schlüssel zum Verschlüsseln und Signaturen prüfen
        \item Privater, geheimer Schlüssel, mit Passwort geschützt
            \begin{itemize}
                \item Entschlüsseln
                \item Signieren
            \end{itemize}
        \item Symmetrische Verschlüsselung der Daten
        \item Asymmetrische Verschlüsselung der symmetrischen Schlüssel
    \end{itemize}
\end{frame}

\begin{frame}{OpenPGP und GnuPG}
    \begin{itemize}
        \item Standard OpenPGP von 1998 (RFC 4880)
        \item GnuPG von Werner Koch (1997)
        \item Portierung auf Microsoft Windows vom BMWA und BMI gefördert (2001/2002)
        \item Standard bei Linux-Distributionen
        \item Verschiedene Front-Ends verfügbar
    \end{itemize}
\end{frame}

\begin{frame}{E-Mail}
    \begin{itemize}
        \item Enigmail für Mozilla Thunderbird
        \item Plugins für viele MUAs
            \begin{itemize}
                \item Claws Mail
                \item Evolution
                \item Microsoft Outlook
                \item KMail
                \item Mutt
                \item Webmailer
            \end{itemize}
        \item \url{https://emailselfdefense.fsf.org/de/}
    \end{itemize}
\end{frame}

\begin{frame}{Web of Trust}
    \begin{itemize}
        \item Alice signiert den Schlüssel von Bob und vertraut Bobs Schlüsselsignaturen
        \item Bob signiert den Schlüssel von Carl
        \item Somit betrachtet Alice den Schlüssel von Carl als gültig
    \end{itemize}
    \pause
    \begin{itemize}
        \item Welche Schlüssel sind vertrauenswürdig?
        \item Keysigning
        \item \url{https://pgp.cs.uu.nl/paths/79be3e4300411886/to/34adcd0072215cc6.html}
    \end{itemize}
\end{frame}

\begin{frame}{Debian}
    \begin{itemize}
        \item Pakete
        \item Metadaten
    \end{itemize}
\end{frame}

\begin{frame}{Git}
    \begin{itemize}
        \item Commits
        \item Tags
        \item \url{https://mikegerwitz.com/papers/git-horror-story}
    \end{itemize}
\end{frame}

\begin{frame}{Weitere Anwendungen}
    \begin{itemize}
        \item Dateien verschlüsseln
            \begin{itemize}
                \item pass (Passwortmanager)
            \end{itemize}
        \item Chat
            \begin{itemize}
                \item XEP-0027, XEP-0373, XEP-0374 für XMPP (Jabber)
            \end{itemize}
    \end{itemize}
\end{frame}

\section*{Solang die dicke Frau noch singt \dots}

\begin{frame}{Kontakt}
    \begin{itemize}
        \item \url{http://www.lespocky.de/}
        \item \url{http://blog.antiblau.de/}
        \item \emph{alex@antiblau.de}
    \end{itemize}

    \vspace{1em}
    \small
    Die Folien sind freigegeben unter \emph{Creative Commons
    Namensnennung-Nicht kommerziell-Weitergabe unter gleichen
    Bedingungen 3.0 Deutschland Lizenz} (BY-NC-SA).
    \normalsize
\end{frame}

\end{document}
