%%%%%%%%%%%%%%%%%%%%%%%%%%%%%%%%%%%%%%%%%%%%%%%%%%%%%%%%%%%%%%%%%%%%%%%%
%%% documentclass and packages
%%%%%%%%%%%%%%%%%%%%%%%%%%%%%%%%%%%%%%%%%%%%%%%%%%%%%%%%%%%%%%%%%%%%%%%%
\RequirePackage{atbegshi}           % workaround for newer PGF versions
%\documentclass[hyperref={pdfpagelabels=false}]{beamer}
%\documentclass[aspectratio=1610,t]{beamer}
\documentclass[t]{beamer}
% https://sourceforge.net/tracker/index.php?func=detail&aid=1848912&group_id=92412&atid=600660
\usepackage{lmodern}
\usepackage[T1]{fontenc}
\usepackage[utf8]{inputenc}
\usepackage{textcomp}

\usepackage[english]{babel}
\usepackage[babel,english=american,german=guillemets]{csquotes}	% french
\usepackage{microtype}
\usepackage{tikz}
\usetikzlibrary{arrows,positioning}
\usepackage{todonotes}
\presetkeys{todonotes}{inline}{}
\usepackage{ulem}

% colors for listings
\definecolor{lightergray}{gray}{.95}
\definecolor{darkblue}{rgb}{0,0,0.5}
\definecolor{darkgreen}{rgb}{0,0.5,0}
\definecolor{darkred}{rgb}{0.5,0,0}
\definecolor{darkerblue}{rgb}{0,0,0.4}
\definecolor{darkergreen}{rgb}{0,0.4,0}
\definecolor{darkerred}{rgb}{0.4,0,0}

\usepackage{listings}
% \lstloadlanguages{HTML,XML}
\lstset{%
    basicstyle=\ttfamily\small\mdseries,
    keywordstyle=\bfseries\color{darkblue},
    identifierstyle=,
    commentstyle=\color{darkgray},
    stringstyle=\itshape\color{darkred},
    frame=none,
    showstringspaces=false,
    tabsize=4,
    backgroundcolor=\color{lightergray},
}

%%%%%%%%%%%%%%%%%%%%%%%%%%%%%%%%%%%%%%%%%%%%%%%%%%%%%%%%%%%%%%%%%%%%%%%%
%%% macros
%%%%%%%%%%%%%%%%%%%%%%%%%%%%%%%%%%%%%%%%%%%%%%%%%%%%%%%%%%%%%%%%%%%%%%%%

% strong emphasis (like in HTML)
\makeatletter
\newcommand{\strong}[1]{\@strong{#1}}
\newcommand{\@@strong}[1]{\textbf{\let\@strong\@@@strong#1}}
\newcommand{\@@@strong}[1]{\textnormal{\let\@strong\@@strong#1}}
\let\@strong\@@strong
\makeatother

% C++ like the C++ FAQ proposes
% https://isocpp.org/wiki/faq/misc-environmental-issues#latex-macros
\newcommand{\CXX}{C\nolinebreak\hspace{-.05em}\raisebox{.4ex}{\tiny\bf +}\nolinebreak\hspace{-.10em}\raisebox{.4ex}{\tiny\bf +}}
\def\CPP{{C\nolinebreak[4]\hspace{-.05em}\raisebox{.4ex}{\tiny\bf ++}}}

%%%%%%%%%%%%%%%%%%%%%%%%%%%%%%%%%%%%%%%%%%%%%%%%%%%%%%%%%%%%%%%%%%%%%%%%
%%% preparations for beamer
%%%%%%%%%%%%%%%%%%%%%%%%%%%%%%%%%%%%%%%%%%%%%%%%%%%%%%%%%%%%%%%%%%%%%%%%
\useinnertheme{default}
\useoutertheme{infolines}
%\usecolortheme[rgb={0.28,0.37,0.52}]{structure}
\usecolortheme[rgb={0.18,0.23,0.33}]{structure}
%\usecolortheme{beaver}
\usefonttheme{structurebold}

%%% Ränder vergrößern für's Café Central
%\setbeamersize{text margin left=1.2cm}
%\setbeamersize{text margin right=1.2cm}

%%% let hyperlinks look like hyperlinks
\hypersetup{%
    colorlinks=true,
    linkcolor=black,
    urlcolor=darkblue
}

%%%%%%%%%%%%%%%%%%%%%%%%%%%%%%%%%%%%%%%%%%%%%%%%%%%%%%%%%%%%%%%%%%%%%%%%
%%% images
%%%%%%%%%%%%%%%%%%%%%%%%%%%%%%%%%%%%%%%%%%%%%%%%%%%%%%%%%%%%%%%%%%%%%%%%
\pgfdeclareimage[width=0.3\textwidth]{cmakelogo}{resources/Cmake.svg.png}
\pgfdeclareimage[height=0.8\textheight]{screenshot-gui}{resources/Screenshot_20181021_141444.png}
\pgfdeclareimage[width=0.6\textwidth]{screenshot-generators}{resources/Screenshot_GUI_Generators.png}
\pgfdeclareimage[width=0.5\textwidth]{screenshot-compiler}{resources/Screenshot_20181021_144601.png}

%%%%%%%%%%%%%%%%%%%%%%%%%%%%%%%%%%%%%%%%%%%%%%%%%%%%%%%%%%%%%%%%%%%%%%%%
%%% title, author, date
%%%%%%%%%%%%%%%%%%%%%%%%%%%%%%%%%%%%%%%%%%%%%%%%%%%%%%%%%%%%%%%%%%%%%%%%
\title[CMake]{Building C/C++ Projects with (Modern) CMake}
\subtitle{Powerful Cross Platform Build Process Management}
\author{Alexander Dahl}
\institute[lespocky.de]{\url{http://www.lespocky.de/}}
\date{2018-10-22}
%\subject{subj}
\keywords{CMake, Compiler, C, C++, Software, Build}

%%%%%%%%%%%%%%%%%%%%%%%%%%%%%%%%%%%%%%%%%%%%%%%%%%%%%%%%%%%%%%%%%%%%%%%%
%%% document
%%%%%%%%%%%%%%%%%%%%%%%%%%%%%%%%%%%%%%%%%%%%%%%%%%%%%%%%%%%%%%%%%%%%%%%%
\begin{document}

\begin{frame}
    \titlepage
\end{frame}

%\begin{frame}
%    \tableofcontents
%\end{frame}


\section*{Who?}

\begin{frame}
    \frametitle{Me}
    \framesubtitle{Yet another free software developer \dots}

    \begin{columns}[T]
        \begin{column}{0.7\textwidth}
            \begin{block}{Background}
                \begin{itemize}
                    \item using Free Software since $\approx 2001$
                    \item contributing to Free Software since $\approx 2003$
                    \item diploma in engineering (mechatronics)
                    \item working as Embedded Linux developer
                    \item member of \href{http://www.netz39.de/}{Netz39}
                        Hackerspace
                \end{itemize}
            \end{block}
        \end{column}
        \pause
        \begin{column}{0.3\textwidth}
            \begin{block}{Projects}
                \begin{itemize}
                    \item \href{https://www.fli4l.de/}{fli4l}
                    \item \href{https://buildroot.org/}{buildroot}
                    \item \href{https://ptxdist.org/}{ptxdist}
                    \item \href{https://github.com/rafaelsteil/libcgi}{libcgi}
                    \item \href{https://freifunk.net/}{Freifunk}
                \end{itemize}
            \end{block}
        \end{column}
    \end{columns}
\end{frame}


\section{CMake for Users}

%\frame{\tableofcontents[currentsection]}

\begin{frame}[fragile]{CMake}
    \begin{columns}[T]
        \begin{column}{0.4\textwidth}
            \begin{itemize}
                \item Build C/\CPP\ projects
                \item Compiler independent
                \item Cross platform
                \item Together with native build environment
                \item Simple configuration with \texttt{CMakeLists.txt}
                    files
                \item Out-of-source builds
                \item Free Software
                \item And more …
            \end{itemize}
        \end{column}
        \begin{column}{0.6\textwidth}
            \pgfuseimage{cmakelogo}
            \begin{lstlisting}[%
                %basicstyle=\ttfamily\small\mdseries
                basicstyle=\ttfamily\footnotesize\mdseries
                %basicstyle=\ttfamily\tiny\mdseries
            ]
cmake_minimum_required(VERSION 3.1)
project(MyProject
    VERSION 1.0
    DESCRIPTION "Very nice project"
    LANGUAGES CXX
)
            \end{lstlisting}
        \end{column}
    \end{columns}
\end{frame}

\begin{frame}{Getting Started}
    \begin{block}{Installation}
        \begin{itemize}
            \item Linux: Use your package manager
            \item Windows, MacOS: Download from \url{https://cmake.org/}
            \item From Source with your favorite \CPP\ Compiler
        \end{itemize}
    \end{block}
    \pause
    \begin{block}{Documentation}
        \begin{itemize}
            \item CMake comes well documented
                \begin{itemize}
                    \item \url{https://cmake.org/documentation}
                    \item \texttt{man 7 cmake-*}
                \end{itemize}
            \item on the world wide web
                \begin{itemize}
                    \item look out for \enquote{Modern CMake}
                    \item beware of examples showing old way to do things
                \end{itemize}
        \end{itemize}
    \end{block}
\end{frame}

\begin{frame}[fragile]{Usage}
    \begin{itemize}
        \item Use out of tree builds
    \end{itemize}
    \begin{block}{Command Line}
        \begin{lstlisting}[%
            basicstyle=\ttfamily\small\mdseries
            %basicstyle=\ttfamily\footnotesize\mdseries
            %basicstyle=\ttfamily\scriptsize\mdseries
            %basicstyle=\ttfamily\tiny\mdseries
        ]
~/path/to/your/src $ mkdir build
~/path/to/your/src $ cd build
~/path/to/your/src/build $ cmake ..
~/path/to/your/src/build $ make
        \end{lstlisting}
    \end{block}
    \begin{itemize}
        \item first call to \texttt{cmake} is special, sets generator
            and compiler
        \item pre-set options with \texttt{-D}
            \begin{itemize}
                \item you can overwrite options again later
                \item useful when building from some external build
                    system\\(like buildroot, ptxdist, …)
            \end{itemize}
    \end{itemize}
\end{frame}

\begin{frame}{GUI}
    \pgfuseimage{screenshot-gui}
\end{frame}

\begin{frame}{Generators}
    \begin{columns}[T]
        \begin{column}{0.4\textwidth}
            \begin{itemize}
                \item Makefiles
                    \begin{itemize}
                        \item Borland
                        \item MSYS
                        \item MinGW
                        \item NMake
                        \item Unix
                        \item …
                    \end{itemize}
                \item Ninja
                \item Visual Studio
                \item Xcode
                \item …
            \end{itemize}
        \end{column}
        \begin{column}{0.6\textwidth}
            \pgfuseimage{screenshot-generators}
        \end{column}
    \end{columns}

    \begin{block}{Help}
        \href{https://cmake.org/cmake/help/latest/manual/cmake-generators.7.html}{cmake-generators(7)}
    \end{block}
\end{frame}

\begin{frame}{Compilers}
    \begin{columns}[T]
        \begin{column}{0.5\textwidth}
            \begin{itemize}
                \item AppleClang
                \item Clang
                \item GNU (GCC)
                \item MSVC (Visual Studio)
                \item SunPro
                \item Intel
                \item XL (IBM)
                \item …
            \end{itemize}
        \end{column}
        \begin{column}{0.5\textwidth}
            \pgfuseimage{screenshot-compiler}
        \end{column}
    \end{columns}

    \begin{block}{Help}
        \href{https://cmake.org/cmake/help/latest/manual/cmake-compile-features.7.html}{cmake-compile-features(7)}
    \end{block}
\end{frame}

\section{CMake for Developers}

%\frame{\tableofcontents[currentsection]}

\subsection{Syntax}

\begin{frame}[fragile]{Syntax}
    \begin{columns}[T]
        \begin{column}{0.6\textwidth}
            \begin{block}{First Command}
                \begin{lstlisting}[%
                    %basicstyle=\ttfamily\small\mdseries
                    basicstyle=\ttfamily\footnotesize\mdseries
                    %basicstyle=\ttfamily\scriptsize\mdseries
                    %basicstyle=\ttfamily\tiny\mdseries
                ]
# Copyright 2018 Jon Doe
cmake_minimum_required(VERSION 3.1)
                \end{lstlisting}
            \end{block}

            \begin{block}{Second Command}
                \begin{lstlisting}[%
                    %basicstyle=\ttfamily\small\mdseries
                    basicstyle=\ttfamily\footnotesize\mdseries
                    %basicstyle=\ttfamily\scriptsize\mdseries
                    %basicstyle=\ttfamily\tiny\mdseries
                ]
project(foo
    VERSION 1.0
    DESCRIPTION "Very nice project"
    LANGUAGES CXX
)
                \end{lstlisting}
            \end{block}

        \end{column}

        \begin{column}{0.4\textwidth}
            \begin{itemize}
                \item One or multiple files called \texttt{CMakeLists.txt}
                \item Commands are documented
                \item Syntax straight forward
                    \begin{itemize}
                        \item Whitespace does\\not matter
                        \item Comments start with '\#' and go to\\end of
                            line
                        \item Variable expansion with
                            \texttt{\$\{VARNAME\}}
                    \end{itemize}
            \end{itemize}
        \end{column}
    \end{columns}

    \begin{block}{Example Command}
        \begin{lstlisting}[%
            %basicstyle=\ttfamily\small\mdseries
            basicstyle=\ttfamily\footnotesize\mdseries
            %basicstyle=\ttfamily\scriptsize\mdseries
            %basicstyle=\ttfamily\tiny\mdseries
        ]
message(STATUS "foo_VERSION: ${foo_VERSION}")
        \end{lstlisting}
    \end{block}
\end{frame}

\subsection{Variables}

\begin{frame}{Variables}
    \begin{itemize}
        \item Set variables with \texttt{set()}
        \item Use \texttt{ALL\_CAPS} for variable names
        \item Access variables with \texttt{\$\{MY\_VARIABLE\}}
        \item Other commands may modify variables
        \item Beware of spaces, if not quoted, that can create lists
        \item Always enclose paths in spaces \texttt{"\$\{MY\_PATH\}"}
        \item Scope is function or directory, not parent (by default)
        \item Variables can be cached across multiple runs
        \item Special global CMake variables exist
    \end{itemize}

    \begin{block}{Help}
        \href{https://cmake.org/cmake/help/latest/manual/cmake-language.7.html}{cmake-language(7)},
        \href{https://cmake.org/cmake/help/latest/manual/cmake-variables.7.html}{cmake-variables(7)}
    \end{block}
\end{frame}

\begin{frame}{Cache and Globals}
    \begin{block}{Cache}
        \begin{itemize}
            \item Cached variables can be listed and manipulated in GUI
            \item Command '\texttt{option()}' is syntactic sugar for
                cached boolean variables
            \item Cache is a text file in build directory
        \end{itemize}
    \end{block}
    \pause
    \begin{block}{Globals}
        \begin{itemize}
            \item \texttt{CMAKE\_BUILD\_TYPE}
%                \begin{description}[RelWithDebInfo]
%                    \item[Debug] no optimization, debug symbols, needed
%                        for gdb
%                    \item[Release] full optimization, no debug symbols
%                    \item[RelWithDebInfo] optimization \emph{and} debug
%                        symbols
%                \end{description}
                \begin{itemize}
                    \item Debug, Release, RelWithDebInfo, …
                \end{itemize}
            \item \texttt{CMAKE\_INSTALL\_PREFIX}
                \begin{itemize}
                    \item defaults to \texttt{/usr/local}, like
                        \texttt{\-\-prefix}
                \end{itemize}
            \item \texttt{BUILD\_SHARED\_LIBS}
        \end{itemize}
    \end{block}
\end{frame}

\subsection{Targets}

\begin{frame}[fragile]{Targets}
    \begin{alertblock}{Modern CMake}
        Think in targets!
    \end{alertblock}

    \pause

    \begin{block}{Simple Executable}
        \begin{lstlisting}[%
            %basicstyle=\ttfamily\small\mdseries
            %basicstyle=\ttfamily\footnotesize\mdseries
            %basicstyle=\ttfamily\scriptsize\mdseries
            %basicstyle=\ttfamily\tiny\mdseries
        ]
add_executable(one
    two.cpp
    three.h
)
target_link_libraries(one
    two::two
)
        \end{lstlisting}
    \end{block}

    \begin{itemize}
        \item Target names
            \begin{itemize}
                \item \texttt{one}
                \item \texttt{two::two}
            \end{itemize}
    \end{itemize}
\end{frame}

\begin{frame}[fragile]{More Targets}
    \begin{block}{Simple Library}
        \begin{lstlisting}[%
            %basicstyle=\ttfamily\small\mdseries
            %basicstyle=\ttfamily\footnotesize\mdseries
            %basicstyle=\ttfamily\scriptsize\mdseries
            %basicstyle=\ttfamily\tiny\mdseries
        ]
add_library(two STATIC
    two.cpp
    three.h
)
add_library(two::two ALIAS two)
        \end{lstlisting}

        \begin{itemize}
            \item Omit \texttt{STATIC} and let \texttt{BUILD\_SHARED\_LIBS}
                decide
            \item Use \texttt{INTERFACE} for header only libraries
        \end{itemize}
    \end{block}

    \begin{block}{Help}
        \href{https://cmake.org/cmake/help/latest/manual/cmake-commands.7.html}{cmake-commands(7)}
    \end{block}
\end{frame}

\begin{frame}[fragile]{Add Build Information}
    \begin{block}{Properties}
        \begin{lstlisting}[%
            %basicstyle=\ttfamily\small\mdseries
            basicstyle=\ttfamily\footnotesize\mdseries
            %basicstyle=\ttfamily\scriptsize\mdseries
            %basicstyle=\ttfamily\tiny\mdseries
        ]
set_target_properties(two PROPERTIES
    SOVERSION   ${PROJECT_VERSION_MAJOR}
    VERSION     ${PROJECT_VERSION}
    C_STANDARD  99
)
        \end{lstlisting}

        \begin{itemize}
%            \item Pieces of information attached to directories,
%                targets, tests, source files, cache entries, and more …
            \item See
                \href{https://cmake.org/cmake/help/latest/manual/cmake-properties.7.html}{cmake-properties(7)}
        \end{itemize}
    \end{block}

    \pause

    \begin{block}{Include Directories}
        \begin{lstlisting}[%
            %basicstyle=\ttfamily\small\mdseries
            basicstyle=\ttfamily\footnotesize\mdseries
            %basicstyle=\ttfamily\scriptsize\mdseries
            %basicstyle=\ttfamily\tiny\mdseries
        ]
target_include_directories(two
    PUBLIC
        $<BUILD_INTERFACE:${PROJECT_SOURCE_DIR}/include>
        $<INSTALL_INTERFACE:${CMAKE_INSTALL_INCLUDEDIR}>
)
        \end{lstlisting}
    \end{block}
\end{frame}

\begin{frame}[fragile]{Glue It Together}
    \begin{alertblock}{The Most Powerful CMake Command}
        \begin{lstlisting}[%
            %basicstyle=\ttfamily\small\mdseries
            %basicstyle=\ttfamily\footnotesize\mdseries
            %basicstyle=\ttfamily\scriptsize\mdseries
            %basicstyle=\ttfamily\tiny\mdseries
        ]
target_link_libraries(one
    PUBLIC
        two::two
    PRIVATE
        ${THREE_LIBRARIES}
)
        \end{lstlisting}
    \end{alertblock}

    \pause

    \begin{block}{Link Items}
        \begin{itemize}
            \item \alert{A library target name}
            \item A full path to a library file
            \item A plain library name
            \item A link flag
            \item A generator expression
        \end{itemize}
    \end{block}
\end{frame}

\begin{frame}{Chain Your Targets}
    \begin{itemize}
        \item Use target names as link items
            \begin{itemize}
                \item Use namespaces, to avoid accidentally wrong
                    interpretation of link items
                \item Use \texttt{PUBLIC}, \texttt{PRIVATE},
                    \texttt{INTERFACE} to not propagate non necessary
                    dependencies
            \end{itemize}
        \item Possible are all kinds of targets\\(even targets, which
            don't lead to actually calling a linker)
        \item CMake generates dependency tree
        \item Build is done in the right order automatically
        \item Use imported targets for external dependencies
        \item Export your own library targets (relocatable packages)
    \end{itemize}
\end{frame}

\subsection{Patterns}

\begin{frame}{Antipatterns}
    \begin{alertblock}{Avoid functions with global scope!}
        \begin{itemize}
            \item e.\,g. \texttt{link\_directories()},
                \texttt{include\_libraries()}
        \end{itemize}
    \end{alertblock}

    \pause

    \begin{alertblock}{Do not set compiler flags in \texttt{CMakeLists.txt}}
        \begin{itemize}
            \item Breaks portability between different compilers
        \end{itemize}
    \end{alertblock}

    \pause

    \begin{exampleblock}{Alternatives}
        \begin{itemize}
            \item Require language standards (e.\,g. C99, \CPP 11, …)
                \begin{itemize}
                    \item Meta compiler features (CMake 3.8+)
                    \item Compiler features (CMake 3.1+)
                    \item Per target properties
                        (\texttt{CXX\_STANDARD}, …)
                \end{itemize}
            \item Manually override compiler flags from
                \enquote{outside} (e.\,g. CMake Cache)
            \item Make use of \texttt{CMAKE\_BUILD\_TYPE}
        \end{itemize}
    \end{exampleblock}
\end{frame}

\begin{frame}{Best Practices}
    \begin{itemize}
        \item Define everything per target
        \item Treat your \texttt{CMakeLists.txt} as code
            \begin{itemize}
                \item Readable
                \item Well documented
                \item Put \texttt{CMakeLists.txt} in version control
            \end{itemize}
        \item Think in targets
        \item Export your targets
        \item Make alias targets (aka namespaces for target names)
            \begin{itemize}
                \item Make use of \texttt{add\_subdirectory()} and
                    \texttt{find\_package()} consistent
                \item Ensure \texttt{target\_link\_libraries()} uses a
                    target
            \end{itemize}
        \item Lower case functions names
        \item Upper case variable names
    \end{itemize}
\end{frame}

\subsection{Imported/Exported Targets and CMake Packages}

\begin{frame}[fragile]{Imported Targets}
    \begin{itemize}
        \item Modern \texttt{find\_package} modules provide
            IMPORTED targets
            \begin{itemize}
                \item e.\,g. \texttt{OpenSSL::SSL}, see
                    \href{https://cmake.org/cmake/help/latest/manual/cmake-modules.7.html}{cmake-modules(7)}
            \end{itemize}
    \end{itemize}

    \begin{lstlisting}[%
        %basicstyle=\ttfamily\small\mdseries
        %basicstyle=\ttfamily\footnotesize\mdseries
        %basicstyle=\ttfamily\scriptsize\mdseries
        %basicstyle=\ttfamily\tiny\mdseries
    ]
find_package(OpenSSL)
target_link_libraries(one
PRIVATE
    OpenSSL::Crypto
)
    \end{lstlisting}

    \begin{itemize}
        \item Use \texttt{find\_package(PkgConfig)} and
            \texttt{pkg\_check\_modules()} with option
            \texttt{IMPORTED\_TARGET}
        \item Create IMPORTED targets with \texttt{add\_library()}
            for your own \texttt{FindFoo.cmake} modules
    \end{itemize}
\end{frame}

\begin{frame}[fragile]{Export Library Targets}
    \begin{lstlisting}[%
        %basicstyle=\ttfamily\small\mdseries
        %basicstyle=\ttfamily\footnotesize\mdseries
        %basicstyle=\ttfamily\scriptsize\mdseries
        %basicstyle=\ttfamily\tiny\mdseries
    ]
include(GNUInstallDirs)				# cmake 2.8.5

install(TARGETS two
	EXPORT two-targets
	LIBRARY DESTINATION "${CMAKE_INSTALL_LIBDIR}"
	ARCHIVE DESTINATION "${CMAKE_INSTALL_LIBDIR}"
)

install(EXPORT two-targets
	NAMESPACE two::
	DESTINATION "${CMAKE_INSTALL_LIBDIR}/cmake/two"
)
    \end{lstlisting}

    \begin{block}{Help}
        \href{https://cmake.org/cmake/help/latest/manual/cmake-packages.7.html}{cmake-packages(7)}
    \end{block}
\end{frame}

\begin{frame}[fragile]{Create (Relocatable) CMake Packages}
    \begin{lstlisting}[%
        %basicstyle=\ttfamily\small\mdseries
        basicstyle=\ttfamily\footnotesize\mdseries
        %basicstyle=\ttfamily\scriptsize\mdseries
        %basicstyle=\ttfamily\tiny\mdseries
    ]
include(CMakePackageConfigHelpers)	# cmake 2.8.8

configure_package_config_file(two-config.cmake.in
	"${CMAKE_CURRENT_BINARY_DIR}/two-config.cmake"
	INSTALL_DESTINATION "${CMAKE_INSTALL_LIBDIR}/cmake/two"
	PATH_VARS CMAKE_INSTALL_INCLUDEDIR
	NO_CHECK_REQUIRED_COMPONENTS_MACRO
)
write_basic_package_version_file(
	"${CMAKE_CURRENT_BINARY_DIR}/two-config-version.cmake"
	COMPATIBILITY SameMajorVersion
)
install(FILES
	"${CMAKE_CURRENT_BINARY_DIR}/two-config.cmake"
	"${CMAKE_CURRENT_BINARY_DIR}/two-config-version.cmake"
	DESTINATION "${CMAKE_INSTALL_LIBDIR}/cmake/two"
)
    \end{lstlisting}

%    \begin{block}{Help}
%        \href{https://cmake.org/cmake/help/latest/manual/cmake-packages.7.html}{cmake-packages(7)}
%    \end{block}
\end{frame}

\section*{What else?}

%\begin{frame}{Legal Stuff}
%\end{frame}

\begin{frame}{The Last Slide}
    \begin{block}{Contact Me}
        \begin{description}[Twitter]
            \item [E-Mail] \href{mailto:post@lespocky.de}{post@lespocky.de}
            \item [WWW] \href{http://www.lespocky.de/}{lespocky.de} or
                    \href{http://blog.antiblau.de/}{blog.antiblau.de}
            \item [Twitter] \href{https://twitter.com/LeSpocky}{@LeSpocky}
        \end{description}
    \end{block}
    \begin{block}{Slides}
        \begin{itemize}
            \item \texttt{hg clone https://bitbucket.org/lespocky/talks}
        \end{itemize}
    \end{block}
    \begin{block}{License}
        These slides are licensed under the Creative Commons
        Attribution-ShareAlike 4.0 International License. (CC BY-SA 4.0) \\
        To view a copy of this license, visit
        \url{http://creativecommons.org/licenses/by-sa/4.0/}.
    \end{block}
\end{frame}

\end{document}
