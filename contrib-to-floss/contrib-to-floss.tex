%%%%%%%%%%%%%%%%%%%%%%%%%%%%%%%%%%%%%%%%%%%%%%%%%%%%%%%%%%%%%%%%%%%%%%%%
%%% documentclass and packages
%%%%%%%%%%%%%%%%%%%%%%%%%%%%%%%%%%%%%%%%%%%%%%%%%%%%%%%%%%%%%%%%%%%%%%%%
\RequirePackage{atbegshi}           % workaround for newer PGF versions
%\documentclass[hyperref={pdfpagelabels=false}]{beamer}
\documentclass{beamer}
% https://sourceforge.net/tracker/index.php?func=detail&aid=1848912&group_id=92412&atid=600660
\usepackage{lmodern}
\usepackage[T1]{fontenc}
\usepackage[utf8]{inputenc}
\usepackage{textcomp}
\usepackage[ngerman]{babel}
\usepackage[babel,english=american,german=guillemets]{csquotes}	% french
\usepackage{microtype}
\usepackage{ulem}

% colors for listings
\definecolor{lightergray}{gray}{.95}
\definecolor{darkblue}{rgb}{0,0,0.5}
\definecolor{darkgreen}{rgb}{0,0.5,0}
\definecolor{darkred}{rgb}{0.5,0,0}
\definecolor{darkerblue}{rgb}{0,0,0.4}
\definecolor{darkergreen}{rgb}{0,0.4,0}
\definecolor{darkerred}{rgb}{0.4,0,0}

%\usepackage{listings}
%\lstloadlanguages{HTML,XML}
%\lstset{
%    basicstyle=\ttfamily\small\mdseries,
%    keywordstyle=\bfseries\color{darkblue},
%    identifierstyle=,
%    commentstyle=\color{darkgray},
%    stringstyle=\itshape\color{darkred},
%    frame=none,
%    showstringspaces=false,
%    tabsize=4,
%    backgroundcolor=\color{lightergray},
%}

%%%%%%%%%%%%%%%%%%%%%%%%%%%%%%%%%%%%%%%%%%%%%%%%%%%%%%%%%%%%%%%%%%%%%%%%
%%% preparations for beamer
%%%%%%%%%%%%%%%%%%%%%%%%%%%%%%%%%%%%%%%%%%%%%%%%%%%%%%%%%%%%%%%%%%%%%%%%
\useinnertheme{default}
\useoutertheme{infolines}
%\usecolortheme[rgb={0.28,0.37,0.52}]{structure}
\usecolortheme[rgb={0.18,0.23,0.33}]{structure}
%\usecolortheme{beaver}
\usefonttheme{structurebold}

%%% Ränder vergrößern für's Café Central
\setbeamersize{text margin left=1.2cm}
\setbeamersize{text margin right=1.2cm}

%%%%%%%%%%%%%%%%%%%%%%%%%%%%%%%%%%%%%%%%%%%%%%%%%%%%%%%%%%%%%%%%%%%%%%%%
%%% images
%%%%%%%%%%%%%%%%%%%%%%%%%%%%%%%%%%%%%%%%%%%%%%%%%%%%%%%%%%%%%%%%%%%%%%%%
%\pgfdeclareimage[width=\textwidth]{lespockydeimpressum}{lespocky.de_impressum}
%\pgfdeclareimage[width=0.6\textwidth]{psituxsmileybeispiel}{psi_tux_smiley_beispiel}

%%%%%%%%%%%%%%%%%%%%%%%%%%%%%%%%%%%%%%%%%%%%%%%%%%%%%%%%%%%%%%%%%%%%%%%%
%%% title, author, date
%%%%%%%%%%%%%%%%%%%%%%%%%%%%%%%%%%%%%%%%%%%%%%%%%%%%%%%%%%%%%%%%%%%%%%%%
\title[Contrib To FLOSS]{Contribute to Free/Libre and OpenSource Software}
\subtitle{HowTo Get Your Things Upstream}
\author{Alexander Dahl}
\institute[lespocky.de]{\url{http://www.lespocky.de/}}
\date{2016-09-26}
\subject{subj}
\keywords{FLOSS, HowTo}

%%%%%%%%%%%%%%%%%%%%%%%%%%%%%%%%%%%%%%%%%%%%%%%%%%%%%%%%%%%%%%%%%%%%%%%%
%%% document
%%%%%%%%%%%%%%%%%%%%%%%%%%%%%%%%%%%%%%%%%%%%%%%%%%%%%%%%%%%%%%%%%%%%%%%%
\begin{document}

\begin{frame}
	\titlepage
\end{frame}

\begin{frame}
    \tableofcontents
\end{frame}


\section*{Who?}

\begin{frame}
    \frametitle{Me}
    \framesubtitle{Yet another free software developer \dots}

    \begin{block}{Background}
        \begin{itemize}
            \item using Free Software since $\approx 2001$
            \item contributing to Free Software since $\approx 2003$
            \item diploma in engineering (mechatronics)
            \item working as Embedded Linux developer
        \end{itemize}
    \end{block}

    \pause

    \begin{block}{Projects}
        \begin{itemize}
            \item fli4l
            \item buildroot
            \item ptxdist
            \item libcgi
            \item Freifunk
        \end{itemize}
    \end{block}
\end{frame}


\section{What?}

\frame{\tableofcontents[currentsection]}

\begin{frame}{Free/Libre and OpenSource Software}
    \begin{block}{What Do We Want?}
        \only<1>
        {
            \begin{itemize}
                \item Free beer!!1!
                \item<2-> Free speech!
            \end{itemize}
        }

        \only<2->
        {
            \begin{itemize}
                \item \sout{Free beer!!1!}
                \item Free speech!
            \end{itemize}
        }
    \end{block}

    \uncover<3->
    {
        \begin{block}{Freedoms as Defined by Free Software Foundation (FSF)}
            \begin{description}
                \item[Freedom 0] The freedom to run the program for any
                    purpose.
                \item[Freedom 1] The freedom to study how the program
                    works, and change it to make it do what you wish.
                \item[Freedom 2] The freedom to redistribute and make
                    copies so you can help your neighbor.
                \item[Freedom 3] The freedom to improve the program, and
                    release your improvements (and modified versions in
                    general) to the public, so that the whole community
                    benefits.
            \end{description}
        \end{block}
    }
\end{frame}

\begin{frame}{Components of FLOSS}
    \begin{itemize}
        \item Sourcecode
        \item Documentation
        \item Artwork
        \item Community
        \item Infrastructure
    \end{itemize}
\end{frame}


\section{Why?}

\frame{\tableofcontents[currentsection]}

\begin{frame}{Why Contribute?}
    \begin{itemize}
        \item add features
        \item fix bugs
        \item improve software
        \item for fun
        \item give back to community
        \item social responsibility
        \item sustainability (use old devices)
        \item learning
        \item business
        \item \dots
    \end{itemize}
\end{frame}

\begin{frame}{Why Upstreaming?}
    \begin{itemize}
        \item more people can use it
        \item review and improvement of your stuff
        \item less future work
        \item avoid fragmentation
        \item \dots
    \end{itemize}
\end{frame}


\section{How?}

\frame{\tableofcontents[currentsection]}

\begin{frame}{It Depends \dots}
    \begin{itemize}
        \item what type of contribution
            \begin{itemize}
                \item Sourcecode
                \item Documentation
                \item Artwork
                \item Community
                \item (Infrastructure)
            \end{itemize}
        \item which project/community
        \item your preference
    \end{itemize}
\end{frame}

\subsection{Communicate}

\begin{frame}{Ask and Answer Questions}
    \framesubtitle{The Community Part}
    \begin{itemize}
        \item mailing list
        \item forum
        \item chat
        \item stackoverflow
        \item social media
        \item usenet
        \item \dots
    \end{itemize}
\end{frame}

\begin{frame}{Talk, Talk, Talk}
    \framesubtitle{It's a trap \dots}
    \begin{itemize}
        \item make yourself familiar with community rules
            \begin{itemize}
                \item Code of Conduct
                \item non written rules
            \end{itemize}
        \item choose the right channel
        \item be polite
        \item be patient
        \item learn.to/quote
    \end{itemize}
\end{frame}

\begin{frame}{Use the Bugtracker}
    \begin{block}{Which}
        \begin{itemize}
            \item Bugzilla (run far, run fast)
            \item Trac
            \item Mantis (maybe run)
            \item Redmine
            \item GitHub
            \item Atlassian Jira (you could run while it still loads)
        \end{itemize}
    \end{block}
    \pause
    \begin{block}{How}
        \begin{itemize}
            \item make sure it's a bug (not a layer 8 problem)
            \item detailed description
            \item reproducible (minimal example)
            \item (add a patch)
        \end{itemize}
    \end{block}
\end{frame}

\subsection{Git}

\begin{frame}{Why Git?}
    \begin{exampleblock}{Why?}
        \begin{itemize}
            \item in 2016 most projects start with Git
            \item GitHub
            \item lots of projects converted their old VCS to Git
            \item distributed (offline) work possible
        \end{itemize}
    \end{exampleblock}
    \pause
    \begin{alertblock}{Why not?}
        \begin{itemize}
            \item the bad things: \url{https://stevebennett.me/2012/02/24/10-things-i-hate-about-git/}
        \end{itemize}
    \end{alertblock}
\end{frame}

\begin{frame}{Contribute Without Git}
    \begin{itemize}
        \item patch and quilt
        \item other VCS
        \item things not in VCS anyway
        \item \dots
    \end{itemize}
\end{frame}

\section*{What else?}

\begin{frame}{The Last Slide}
    \begin{block}{How To Reach Me}
        \begin{itemize}
            \item @LeSpocky
            \item \url{http://www.lespocky.de/}
            \item \url{http://blog.antiblau.de/}
            \item \emph{post@lespocky.de}
        \end{itemize}
    \end{block}
    \begin{block}{License}
        These slides are licensed under the Creative Commons
        Attribution-ShareAlike 4.0 International License. (CC BY-SA 4.0) \\
        To view a copy of this license, visit
        \url{http://creativecommons.org/licenses/by-sa/4.0/}.
    \end{block}
\end{frame}

\end{document}
