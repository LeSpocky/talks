%%%%%%%%%%%%%%%%%%%%%%%%%%%%%%%%%%%%%%%%%%%%%%%%%%%%%%%%%%%%%%%%%%%%%%%%
%%% documentclass and packages
%%%%%%%%%%%%%%%%%%%%%%%%%%%%%%%%%%%%%%%%%%%%%%%%%%%%%%%%%%%%%%%%%%%%%%%%
\RequirePackage{atbegshi}           % workaround for newer PGF versions
%\documentclass[hyperref={pdfpagelabels=false}]{beamer}
\documentclass{beamer}
% https://sourceforge.net/tracker/index.php?func=detail&aid=1848912&group_id=92412&atid=600660
\usepackage{lmodern}
\usepackage[T1]{fontenc}
\usepackage[utf8]{inputenc}
\usepackage{textcomp}
\usepackage[ngerman]{babel}
\usepackage[babel,english=american,german=guillemets]{csquotes}	% french
\usepackage{microtype}
\usepackage{ulem}

% colors for listings
\definecolor{lightergray}{gray}{.95}
\definecolor{darkblue}{rgb}{0,0,0.5}
\definecolor{darkgreen}{rgb}{0,0.5,0}
\definecolor{darkred}{rgb}{0.5,0,0}
\definecolor{darkerblue}{rgb}{0,0,0.4}
\definecolor{darkergreen}{rgb}{0,0.4,0}
\definecolor{darkerred}{rgb}{0.4,0,0}

%\usepackage{listings}
%\lstloadlanguages{HTML,XML}
%\lstset{
%    basicstyle=\ttfamily\small\mdseries,
%    keywordstyle=\bfseries\color{darkblue},
%    identifierstyle=,
%    commentstyle=\color{darkgray},
%    stringstyle=\itshape\color{darkred},
%    frame=none,
%    showstringspaces=false,
%    tabsize=4,
%    backgroundcolor=\color{lightergray},
%}

%%%%%%%%%%%%%%%%%%%%%%%%%%%%%%%%%%%%%%%%%%%%%%%%%%%%%%%%%%%%%%%%%%%%%%%%
%%% preparations for beamer
%%%%%%%%%%%%%%%%%%%%%%%%%%%%%%%%%%%%%%%%%%%%%%%%%%%%%%%%%%%%%%%%%%%%%%%%
\useinnertheme{default}
\useoutertheme{infolines}
%\usecolortheme[rgb={0.28,0.37,0.52}]{structure}
\usecolortheme[rgb={0.18,0.23,0.33}]{structure}
%\usecolortheme{beaver}
\usefonttheme{structurebold}

%%% Ränder vergrößern für's Café Central
\setbeamersize{text margin left=1.2cm}
\setbeamersize{text margin right=1.2cm}

%%%%%%%%%%%%%%%%%%%%%%%%%%%%%%%%%%%%%%%%%%%%%%%%%%%%%%%%%%%%%%%%%%%%%%%%
%%% images
%%%%%%%%%%%%%%%%%%%%%%%%%%%%%%%%%%%%%%%%%%%%%%%%%%%%%%%%%%%%%%%%%%%%%%%%
%\pgfdeclareimage[width=\textwidth]{lespockydeimpressum}{lespocky.de_impressum}
%\pgfdeclareimage[width=0.6\textwidth]{psituxsmileybeispiel}{psi_tux_smiley_beispiel}

%%%%%%%%%%%%%%%%%%%%%%%%%%%%%%%%%%%%%%%%%%%%%%%%%%%%%%%%%%%%%%%%%%%%%%%%
%%% title, author, date
%%%%%%%%%%%%%%%%%%%%%%%%%%%%%%%%%%%%%%%%%%%%%%%%%%%%%%%%%%%%%%%%%%%%%%%%
\title[Contrib To FLOSS]{Contribute to Free/Libre and OpenSource Software}
\subtitle{HowTo Get Your Things Upstream}
\author{Alexander Dahl}
\institute[lespocky.de]{\url{http://www.lespocky.de/}}
\date{2016-09-26}
\subject{subj}
\keywords{FLOSS, HowTo}

%%%%%%%%%%%%%%%%%%%%%%%%%%%%%%%%%%%%%%%%%%%%%%%%%%%%%%%%%%%%%%%%%%%%%%%%
%%% document
%%%%%%%%%%%%%%%%%%%%%%%%%%%%%%%%%%%%%%%%%%%%%%%%%%%%%%%%%%%%%%%%%%%%%%%%
\begin{document}

\begin{frame}
	\titlepage
\end{frame}

%\begin{frame}{Überblick}
%    \tableofcontents
%\end{frame}

\section*{Who?}

\begin{frame}
    \frametitle{Me}
    \framesubtitle{Yet another free software developer \dots}

    \begin{block}{Background}
        \begin{itemize}
            \item using Free Software since $\approx 2001$
            \item contributing to Free Software since $\approx 2003$
            \item diploma in engineering (mechatronics)
            \item working as Embedded Linux developer
        \end{itemize}
    \end{block}

    \pause

    \begin{block}{Projects}
        \begin{itemize}
            \item fli4l
            \item buildroot
            \item ptxdist
            \item libcgi
            \item Freifunk
        \end{itemize}
    \end{block}
\end{frame}

\section{What?}

\begin{frame}{Free/Libre and OpenSource Software}
    \begin{block}{What Do We Want?}
        \only<1>
        {
            \begin{itemize}
                \item Free beer!!1!
                \item<2-> Free speech!
            \end{itemize}
        }

        \only<2->
        {
            \begin{itemize}
                \item \sout{Free beer!!1!}
                \item Free speech!
            \end{itemize}
        }
    \end{block}

    \uncover<3->
    {
        \begin{block}{Freedoms as Defined by Free Software Foundation (FSF)}
            \begin{description}
                \item[Freedom 0] The freedom to run the program for any
                    purpose.
                \item[Freedom 1] The freedom to study how the program
                    works, and change it to make it do what you wish.
                \item[Freedom 2] The freedom to redistribute and make
                    copies so you can help your neighbor.
                \item[Freedom 3] The freedom to improve the program, and
                    release your improvements (and modified versions in
                    general) to the public, so that the whole community
                    benefits.
            \end{description}
        \end{block}
    }
\end{frame}

\begin{frame}{Components of FLOSS}
    \begin{itemize}
        \item Sourcecode
        \item Documentation
        \item Artwork
        \item Community
        \item Infrastructure
    \end{itemize}
\end{frame}

\section{Why?}

\begin{frame}{Why contribute?}
    \begin{itemize}
        \item Fun
        \item Give back
        \item Business
        \item \dots
    \end{itemize}
\end{frame}

\begin{frame}{Why upstreaming?}
    \begin{itemize}
        \item more people can use it
        \item review and improvement of your stuff
        \item less future work
        \item \dots
    \end{itemize}
\end{frame}

\section{How?}

\begin{frame}{It depends \dots}
    \begin{itemize}
        \item what type of contribution
        \item which project/community
        \item your preference
        \item \dots
    \end{itemize}
\end{frame}

\subsection{Communicate}

\subsection{Git}

\section*{What else?}

\begin{frame}{The Last Slide}
    \begin{block}{How To Reach Me}
        \begin{itemize}
            \item @LeSpocky
            \item \url{http://www.lespocky.de/}
            \item \url{http://blog.antiblau.de/}
            \item \emph{post@lespocky.de}
        \end{itemize}
    \end{block}
    \begin{block}{License}
        These slides are licensed under the Creative Commons
        Attribution-ShareAlike 4.0 International License. (CC BY-SA 4.0) \\
        To view a copy of this license, visit
        \url{http://creativecommons.org/licenses/by-sa/4.0/}.
    \end{block}
\end{frame}

\end{document}
