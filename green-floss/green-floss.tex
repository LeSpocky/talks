%%%%%%%%%%%%%%%%%%%%%%%%%%%%%%%%%%%%%%%%%%%%%%%%%%%%%%%%%%%%%%%%%%%%%%%%
%%% documentclass and packages
%%%%%%%%%%%%%%%%%%%%%%%%%%%%%%%%%%%%%%%%%%%%%%%%%%%%%%%%%%%%%%%%%%%%%%%%
\RequirePackage{atbegshi}           % workaround for newer PGF versions
%\documentclass[hyperref={pdfpagelabels=false}]{beamer}
%\documentclass[aspectratio=1610,t]{beamer}
\documentclass[t]{beamer}
% https://sourceforge.net/tracker/index.php?func=detail&aid=1848912&group_id=92412&atid=600660
\usepackage{lmodern}
\usepackage[T1]{fontenc}
\usepackage[utf8]{inputenc}
\usepackage{textcomp}

\usepackage[english]{babel}
\usepackage[babel,english=american,german=guillemets]{csquotes}	% french
\usepackage{microtype}
\usepackage{tikz}
\usetikzlibrary{arrows,positioning}
\usepackage{smartdiagram}
\usepackage{todonotes}
\presetkeys{todonotes}{inline}{}
\usepackage{ulem}

% colors for listings
\definecolor{lightergray}{gray}{.95}
\definecolor{darkblue}{rgb}{0,0,0.5}
\definecolor{darkgreen}{rgb}{0,0.5,0}
\definecolor{darkred}{rgb}{0.5,0,0}
\definecolor{darkerblue}{rgb}{0,0,0.4}
\definecolor{darkergreen}{rgb}{0,0.4,0}
\definecolor{darkerred}{rgb}{0.4,0,0}

\usepackage{listings}
% \lstloadlanguages{HTML,XML}
\lstset{%
    basicstyle=\ttfamily\small\mdseries,
    keywordstyle=\bfseries\color{darkblue},
    identifierstyle=,
    commentstyle=\color{darkgray},
    stringstyle=\itshape\color{darkred},
    frame=none,
    showstringspaces=false,
    tabsize=4,
    backgroundcolor=\color{lightergray},
}

%%%%%%%%%%%%%%%%%%%%%%%%%%%%%%%%%%%%%%%%%%%%%%%%%%%%%%%%%%%%%%%%%%%%%%%%
%%% macros
%%%%%%%%%%%%%%%%%%%%%%%%%%%%%%%%%%%%%%%%%%%%%%%%%%%%%%%%%%%%%%%%%%%%%%%%

% strong emphasis (like in HTML)
\makeatletter
\newcommand{\strong}[1]{\@strong{#1}}
\newcommand{\@@strong}[1]{\textbf{\let\@strong\@@@strong#1}}
\newcommand{\@@@strong}[1]{\textnormal{\let\@strong\@@strong#1}}
\let\@strong\@@strong
\makeatother

% C++ like the C++ FAQ proposes
% https://isocpp.org/wiki/faq/misc-environmental-issues#latex-macros
\newcommand{\CXX}{C\nolinebreak\hspace{-.05em}\raisebox{.4ex}{\tiny\bf +}\nolinebreak\hspace{-.10em}\raisebox{.4ex}{\tiny\bf +}}
\def\CPP{{C\nolinebreak[4]\hspace{-.05em}\raisebox{.4ex}{\tiny\bf ++}}}

%%%%%%%%%%%%%%%%%%%%%%%%%%%%%%%%%%%%%%%%%%%%%%%%%%%%%%%%%%%%%%%%%%%%%%%%
%%% preparations for beamer
%%%%%%%%%%%%%%%%%%%%%%%%%%%%%%%%%%%%%%%%%%%%%%%%%%%%%%%%%%%%%%%%%%%%%%%%
\useinnertheme{default}
\useoutertheme{infolines}
%\usecolortheme[rgb={0.28,0.37,0.52}]{structure}
\usecolortheme[rgb={0.18,0.23,0.33}]{structure}
%\usecolortheme{beaver}
\usefonttheme{structurebold}

%%% Ränder vergrößern für's Café Central
%\setbeamersize{text margin left=1.2cm}
%\setbeamersize{text margin right=1.2cm}

%%% let hyperlinks look like hyperlinks
\hypersetup{%
    colorlinks=true,
    linkcolor=black,
    urlcolor=darkblue
}

%%%%%%%%%%%%%%%%%%%%%%%%%%%%%%%%%%%%%%%%%%%%%%%%%%%%%%%%%%%%%%%%%%%%%%%%
%%% images
%%%%%%%%%%%%%%%%%%%%%%%%%%%%%%%%%%%%%%%%%%%%%%%%%%%%%%%%%%%%%%%%%%%%%%%%
%\pgfdeclareimage[height=0.8\paperheight]{sdg}{sdg}
\pgfdeclareimage[width=\textwidth]{sdg}{sdg}
\pgfdeclareimage[width=0.18\textwidth]{fp2}{resources/fairphone/webshop_1428x1428_indigo.jpg}
\pgfdeclareimage[width=0.48\textwidth]{rockbox}{resources/rockbox/ipodnano2g-front.png}
\pgfdeclareimage[width=0.3\textwidth]{wdr4300}{resources/openwrt/img_4361.jpg}

%%%%%%%%%%%%%%%%%%%%%%%%%%%%%%%%%%%%%%%%%%%%%%%%%%%%%%%%%%%%%%%%%%%%%%%%
%%% title, author, date
%%%%%%%%%%%%%%%%%%%%%%%%%%%%%%%%%%%%%%%%%%%%%%%%%%%%%%%%%%%%%%%%%%%%%%%%
\title[Green FLOSS]{Freie Software als Beitrag zu Green IT}
\subtitle{Eine nachhaltigere und umweltfreundlichere Informationstechnologie ist möglich …}
\author{Alexander Dahl}
\institute[blog.antiblau.de]{\url{http://blog.antiblau.de/}}
\date{2019-04-29}
%\subject{subj}
%\keywords{FLOSS}

%%%%%%%%%%%%%%%%%%%%%%%%%%%%%%%%%%%%%%%%%%%%%%%%%%%%%%%%%%%%%%%%%%%%%%%%
%%% document
%%%%%%%%%%%%%%%%%%%%%%%%%%%%%%%%%%%%%%%%%%%%%%%%%%%%%%%%%%%%%%%%%%%%%%%%
\begin{document}

\begin{frame}
    \titlepage
\end{frame}

%\begin{frame}
%    \tableofcontents
%\end{frame}


\section*{Wer?}

\begin{frame}
    \frametitle{Me}
    \framesubtitle{Yet another free software developer \dots}

    \begin{columns}[T]
        \begin{column}{0.7\textwidth}
            \begin{block}{Background}
                \begin{itemize}
                    \item benutzt Freie Software seit $\approx 2001$
                    \item trägt zu Freier Software bei seit $\approx 2003$
                    \item Diplom-Ingenieur (Mechatronik)
                    \item arbeitet als Embedded Software Entwickler
                    \item Mitglied des \href{http://www.netz39.de/}{Netz39} Hackerspace
                \end{itemize}
            \end{block}
        \end{column}
        \pause
        \begin{column}{0.3\textwidth}
            \begin{block}{Projekte}
                \begin{itemize}
                    \item \href{https://www.fli4l.de/}{fli4l}
                    %\item \href{https://buildroot.org/}{buildroot}
                    \item \href{https://ptxdist.org/}{ptxdist}
                    \item \href{https://github.com/rafaelsteil/libcgi}{libcgi}
                    %\item \href{https://freifunk.net/}{Freifunk}
                    \item \href{https://www.kernel.org/}{Linux}
                    \item \href{https://www.denx.de/wiki/U-Boot/}{U-Boot}
                \end{itemize}
            \end{block}
        \end{column}
    \end{columns}
\end{frame}

\section{Was}

\begin{frame}{F(L)OSS -- Free/Libre and OpenSource Software}
    \begin{block}{Freie Software nach Definition von GNU, FSF, FSFE}
        \begin{description}
            \item[Freiheit 1] \strong{Verwenden.} {\small Die Freiheit,
                das Programm auszuführen wie man möchte, für jeden
                Zweck.}
            \item[Freiheit 2] \strong{Verstehen.} {\small Die Freiheit,
                die Funktionsweise eines Programms zu untersuchen, und
                es an seine Bedürfnisse anzupassen.}
            \item[Freiheit 3] \strong{Verbreiten.} {\small Die Freiheit,
                Kopien weiterzugeben und damit seinen Mitmenschen zu
                helfen.}
            \item[Freiheit 4] \strong{Verbessern.} {\small Die Freiheit,
                ein Programm zu verbessern, und die Verbesserungen
                an die Öffentlichkeit weiterzugeben, sodass die
                gesamte Gesellschaft profitiert.}
        \end{description}
    \end{block}
    \pause
    \begin{block}{Open Source Software}
        \begin{itemize}
            \item Freizügigere Lizenzen (permissive licenses)
        \end{itemize}
    \end{block}
\end{frame}

\begin{frame}[c]{Green IT}
    \begin{centering}
            \smartdiagramset{%
                bubble center node size=3.8cm,
                bubble center node color=green!20,
                bubble node size=3cm,
                bubble node font=\bfseries\Large,
            }
            \smartdiagram[bubble diagram]{%
                ,Ressourcen,Langlebigkeit,Energie
            }
        \par
    \end{centering}
\end{frame}

\begin{frame}{Agenda 2030 -- Ziele für nachhaltige Entwicklung}
    \pgfuseimage{sdg}
\end{frame}

\section{Wie}

\subsection{Wissen}

\begin{frame}{Forschung und Untersuchungen}
    \begin{block}{Bundesministerium für Umwelt, Naturschutz und nukleare Sicherheit (BMU)}
        \begin{itemize}
            \item \href{https://www.bmu.de/themen/forschung-foerderung/forschung/forschungs-und-entwicklungsberichte/details/ermittlung-und-erschliessung-von-umweltschutzpotenzialen-der-informations-und-kommunikationstechnik/}{Ermittlung und Erschließung von Umweltschutzpotenzialen der Informations- und Kommunikationstechnik (Green IT)…} (2015, Abschnitt 3.1.6 \enquote{Die Rolle von Open Source Software})
        \end{itemize}
    \end{block}
    \begin{block}{Green Software Engineering}
        \begin{itemize}
            \item \url{http://green-software-engineering.de/}
            \item Forschungsbereich \enquote{Grüne Software} am Umwelt-Campus Birkenfeld der Hochschule Trier
            \item lange Liste wissenschaftlicher Veröffentlichungen
            \item Kriterienkatalog
        \end{itemize}
    \end{block}
\end{frame}

\begin{frame}{Ansätze}
    \begin{itemize}
        \item Energieverbrauch von IT-Systemen selbst reduzieren
        \item Ressourcenverbrauch von IT-Systemen reduzieren \\
            (CPU, RAM, Storage)
        \item Lebensdauer verlängern
        \item Reduktion von Ressourcen- und Energieverbrauch in anderen Bereichen \\
            (Verkehr, Heizung, Solaranlagen, …)
        \item Vermeidung von unnötigen Parallelentwicklungen
    \end{itemize}
\end{frame}

\subsection{Projekte}

\begin{frame}{Fairphone}
    \begin{columns}[T]
        \begin{column}{0.77\textwidth}
            \begin{itemize}
                \item Herstellung von Smartphones problematisch
                    \begin{itemize}
                        \item Rohstoffe aus Krisen- und Kriegsgebieten
                        \item Arbeitsbedingungen in Herstellerbetrieben
                        \item Langlebigkeit, Reparaturfähigkeit und
                            Recycling?
                    \end{itemize}
            \end{itemize}
        \end{column}
        \begin{column}{0.2\textwidth}
            \pgfuseimage{fp2}
        \end{column}
    \end{columns}
    \begin{itemize}
        \item erklärtes Ziel: Langlebigkeit durch Unterstützung Freier
            Software
        \item FP1 kam aus verschiedenen Gründen nicht über Android 4.2
            hinaus
        \item FP2 ab Weihnachten 2015 ausgeliefert
        \item FairphoneOS 18.09 mit Android 7.1.2 (Nougat) ab November
            2018
        \item \href{https://wiki.lineageos.org/devices/FP2}{LineageOS 15.1}
            mit Android 8.1 (Oreo) ab August 2018
        \item beste Android Security zusammen mit einigen Google
            Geräten\footnote{\href{https://twitter.com/SecX13/status/1115380640298487808}{@SecX13} on Twitter}
        \item langer Android Support setzt freien Hardware Support
            voraus
    \end{itemize}
\end{frame}

\begin{frame}{Rockbox}
    \begin{columns}[T]
        \begin{column}{0.45\textwidth}
            \begin{itemize}
                \item freies Betriebssystem für MP3-Player unter GPL
                \item gestartet 2002 für Archos Jukebox
                \item Support für ca. 40 Devices, auch einige Apple iPod
                \item Unterstützung zusätzlicher Audio-Codecs
                \item Port auf Android in Arbeit
            \end{itemize}
        \end{column}
        \begin{column}{0.5\textwidth}
            \pgfuseimage{rockbox}
        \end{column}
    \end{columns}
\end{frame}

\begin{frame}{Linux Router}
    \begin{itemize}
        \item OpenWRT, fli4l, IPFire, …
            \begin{itemize}
                \item siehe bspw.  \href{https://en.wikipedia.org/wiki/List_of_router_and_firewall_distributions}{List of router and firewall distributions}
            \end{itemize}
        \item Support von Router-Herstellern für ältere Modelle endet
            schnell
        \item letzte Firmware von TP-Link für TL-WDR4300 von
            2015\footnote{\tiny\url{https://www.tp-link.com/de/support/download/tl-wdr4300/\#Firmware}}
    \end{itemize}
    \begin{columns}[b]
        \begin{column}{0.6\textwidth}
            \begin{block}{OpenWRT}
                \begin{itemize}
                    \item Linux-Distribution für \enquote{Plasterouter}
                    \item viele Geräte unterstützt
                    \item aktiv gepflegt
                    \item vollständige Router-Funktionen inkl. DNS/DHCP,
                        WiFi, VLAN, VPN, …
                    \item Zugriff über SSH oder Weboberfläche
                \end{itemize}
            \end{block}
        \end{column}
        \begin{column}{0.35\textwidth}
            \pgfuseimage{wdr4300}
        \end{column}
    \end{columns}
\end{frame}

\begin{frame}{Weitere Projekte, Institutionen und Initiativen}
    \begin{block}{Developers for Future}
        \begin{itemize}
            \item \url{https://developersforfuture.org/}
            \item
                \href{https://twitter.com/DevsForFuture}{@DevsForFuture}
                bei Twitter
            \item Unterstützung für FridaysForFuture
            \item Bildungsarbeit
            \item Nachhaltige Softwareentwicklung (Open Source)
        \end{itemize}
    \end{block}
\end{frame}

\section*{Was noch?}

\begin{frame}{Die vorletzte Folie}
    \begin{block}{Kontakt}
        \begin{description}[Twitter]
            \item [E-Mail] \href{mailto:post@lespocky.de}{post@lespocky.de}
            \item [WWW] \href{http://www.lespocky.de/}{lespocky.de} or
                    \href{http://blog.antiblau.de/}{blog.antiblau.de}
            \item [Twitter] \href{https://twitter.com/LeSpocky}{@LeSpocky}
        \end{description}
    \end{block}
    \begin{block}{Folien}
        \begin{itemize}
            \item \texttt{hg clone https://bitbucket.org/lespocky/talks}
        \end{itemize}
    \end{block}
    \begin{block}{Lizenz}
        Dieses Werk ist lizenziert unter einer
        \href{http://creativecommons.org/licenses/by-sa/4.0/}{Creative Commons
        Namensnennung - Weitergabe unter gleichen Bedingungen 4.0 International
        Lizenz}.
    \end{block}
\end{frame}

\begin{frame}{Lizenzen}
    \begin{block}{The Global Goals}
        See globalgoals.org
        \href{https://www.globalgoals.org/asset-licence}{Asset Licence}.
    \end{block}
    \begin{block}{Fairphone}
        Bild des \href{https://shop.fairphone.com/de/}{Fairphone 2} ist
        lizensiert unter
        \href{https://creativecommons.org/licenses/by-nc-sa/4.0/}{Creative
        Commons Namensnennung - Nicht-kommerziell - Weitergabe unter
        gleichen Bedingungen 4.0 International Lizenz}.
    \end{block}
    \begin{block}{Rockbox}
        Bild des iPod Nano aus der Dokumentation von Rockbox unter
        \href{https://download.rockbox.org/daily/manual/rockbox-ipodnano2g/rockbox-buildap9.html}{GNU Free Documentation License}.
    \end{block}
\end{frame}

\end{document}
